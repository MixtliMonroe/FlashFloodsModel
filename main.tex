\documentclass[12pt]{article}
\usepackage[hidelinks]{hyperref}
\usepackage{tikz}
\usepackage{float}
\usepackage{amsmath, amssymb, amsfonts}
\usepackage{cancel}
\usepackage{subcaption}
\usepackage{graphicx} % Required for inserting images

\usepackage{geometry} % Define margins
 \geometry{
 a4paper,
 total={170mm,257mm},
 left=20mm,
 right=20mm,
 top=20mm,
 }

\title{Modelling Flash Floods}
\author{Mixtli Monroe}
\date{February 2026}

\begin{document}

\maketitle

\vspace{0.1\textheight}

\begin{abstract}
\begin{center}
    Abstract
\end{center}
\end{abstract}

\tableofcontents

\pagebreak

\section{Introduction}

\section{Physical Model}

\section{Nondimensionalisation}
Equation \eqref{} suggests the height of the river from its deepest point $h$ as a function of the river's width $w$ follows:
\begin{align*}
    h(w) = h_0 (w/w_0)^n
\end{align*}
where $h_0$ and $w_0$ are defined to be the unique values where $w=w_0\iff h=h_0$ for all $n$.\\
This naturally leads to the relabelling $h\rightarrow \tilde{h}=:h_0h$ and $w\rightarrow \tilde{w}=:w_0w$ where the height and width of the river can now be measured with dimensionless quantities $h,w$ as multiples of the characteristic height and width $h_0,w_0$.\\
The cross-sectional area is thus:
\begin{align*}
    A=h_0w_0\frac{n}{n+1}h^{\frac{n+1}{n}}
\end{align*}
Which also naturally suggests relabelling to measure the cross-sectional area in units of $h_0w_0$: $A\rightarrow \tilde{A}=:h_0w_0A$.\\
So far, we have the relations:
\begin{align*}
    h &= w^n \tag{5}\label{eq5}\\
    A &= \frac{n}{n+1}h^{\frac{n+1}{n}} \tag{6}\label{eq6}
\end{align*}
Substituting these into the PDE \eqref{} yields:
\begin{align*}
    \frac{\partial A}{\partial t}+\sqrt{\frac{gh_0w_0\sin\alpha}{f}}\frac{\partial}{\partial s}\left[\sqrt{\frac{A^3}{L[\tilde{h}]}}\ \right]=0
\end{align*}
The remaining dimensionfull quantities in this equation are $s,t,$ and $L$ with the wetted perimeter $L$ being the least trivial to nondimensionalise.\\
To do so, consider the explicit integral form (in terms of the half-width $\tilde{w}/2$):
\begin{align*}
    L &= 2\int_0^{\tilde{w}/2}\sqrt{1+\left(\frac{d\tilde{h}}{d[\tilde{w}'/2]}\right)^2}\ d[\tilde{w}'/2]\\
    &=w_0\ 2\int_0^{w/2}\sqrt{1+\left(\lambda \frac{dh}{d[w'/2]}\right)^2}\ d[w'/2]
\end{align*}
where the aspect ratio $\lambda$ is defined as $\lambda:=h_0/w_0$. Notice that $\lambda$ cannot be factored out, and so is a necessary physical parameter.\\
This naturally suggests measuring $L$ in units of $w_0$, so relabelling $L[\tilde{h}]\rightarrow\tilde{L}[\tilde{h}]=:w_0 L[\lambda]$ results in the PDE taking the form:
\begin{align*}
    \frac{\partial A}{\partial t}+\sqrt{\frac{gh_0\sin\alpha}{f}}\frac{\partial}{\partial s}\left[\sqrt{\frac{A^3}{L[\lambda]}}\ \right]=0
\end{align*}
Notice that the term $\sqrt{gh_0\sin(\alpha)/f}$ has units of velocity. This can be used to define some characteristic velocity $v_c:=\sqrt{gh_0\sin(\alpha)/f}$ so that, in relabelling $s\rightarrow\tilde{s}=:Ss,t\rightarrow\tilde{t}=:Tt$ to measure with dimensionless quantities $s$ and $t$, we can choose $S=v_cT$ for the PDE to take the simple form:
\begin{align*}
    \frac{\partial A}{\partial t} + \frac{\partial}{\partial s}\left[\sqrt{\frac{A^3}{L[\lambda]}}\ \right]=0 \tag{7}\label{eq7}
\end{align*}
Notice that $v_c$ aligns nicely with the definition of $\bar{u}$, providing a unit for area-averaged fluid velocity too if need be.\\
To conclude the nondimensionalisation process, we have:
\begin{align*}
    h:=\tilde{h}/h_0,\ &w:=\tilde{w}/w_0,\\
    A:=\tilde{A}/(h_0w_0),\ &L:=\tilde{L}/w_0,\\
    s:=\tilde{s}/S,\ &t:=\tilde{t}/(v_cS)\\
    \text{where }v_c:=&\sqrt{gh_0\sin(\alpha)/f}\tag{8}\label{eq8}\\
\end{align*}




\section{Method of Characteristics}
Equation \eqref{eq6}, although compact, is not a very useful form


\end{document}